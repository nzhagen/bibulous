\documentclass{article}
\usepackage[utf8]{inputenc} %File encoding
\usepackage{ifthen}             %% needed for the redefinition of \@cite below.

\newcommand{\citename}[1]{#1\protect\nocite{#1}}

\renewenvironment{glossary}
   {\begin{list}{}{\setlength\labelsep{\linewidth}%
                   \setlength\labelwidth{0pt}%
                   \setlength\itemindent{-\leftmargin}%
                   \let\makelabel\descriptionlabel}}
   {\end{list}}

\makeatletter
   \renewcommand\@biblabel[1]{#1}
   \renewenvironment{thebibliography}[1]
   {\section*{\refname}%
       \list{}{\setlength\labelwidth{1.5cm}%
          \leftmargin\labelwidth \advance\leftmargin\labelsep
          \let\makelabel\descriptionlabel}}%
   {\endlist}
   \renewcommand{\@cite}[2]{{#1\ifthenelse{\boolean{@tempswa}}{,\nolinebreak[3] #2}{}}}
\makeatother


\begin{document}

Our network uses \citename{AD}. By using \citename{AD} with \citename{JS} bases clients that have been installed using a \citename{RF} from \citename{CD}, we can expect a high level of standardization.

If you calculate with \cite{sym:pi} you always get an irrational result, because \cite{sym:pi} itself is irrational. As a matter of fact, there are \cite{sym:phi} and \cite{sym:lambda}, too.

\bibliographystyle{test8}
\bibliography{test8}


\end{document}